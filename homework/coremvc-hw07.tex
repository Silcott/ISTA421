\documentclass{article}

    \usepackage[margin=1in]{geometry}
    \usepackage{hyperref}
    \usepackage{listings}
    \usepackage{graphics}
    \usepackage{pgf}
    \usepackage{amsmath}
    \lstset{
        breaklines = true,
        language = [Sharp]C,
        numbers = left,
        basicstyle = \footnotesize
    }
    \usepackage{fancyhdr}
    \pagestyle{fancy}
    \fancyfoot[c]{\footnotesize Page \thepage, Revised on \today{} by Charles Carter}
    \renewcommand{\headrulewidth}{0pt}
    \renewcommand{\footrulewidth}{0.5pt}
    \newcommand{\asp}{\textsf{ASP.NET Core MVC  }}
    

    \title{ASP.NET Programming Homework 07}
    \author{Chapter 07, Pro ASP.NET Core MVC 2}
    \date{}

\begin{document}    

    \maketitle{}
    \thispagestyle{fancy}

    \section*{Homework}

        \subsection*{Readings}

        Read chapter 07, in the \textit{Pro ASP.NET Core MVC 2} book.
        
        \subsection*{Discussion Questions}

        \paragraph{Summary} From your reading, select five major points that you think are the most important points covered in the chapter. For each point, write a summary of the topic. \textit{Summary} means a good explanation, not just a few words or one sentence, but also not a three page paper. A good paragraph is sufficient. Write your summary as if you wanted to explain the topic to a good friend or your grandmother. 

        \paragraph{Question} After you complete your summary, write a concise question that will elicit a response indicating that the person answering the question understands the topic. The idea is that your question could be a test question on an examination, and that your summary could be the ``correct'' answer.

        \paragraph{Format} Format your homework in MarkDown. Add and commit it to your local git repository, and push the homework up to your Github repository.
\end{document}    
